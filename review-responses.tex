%%%%%%
%%
%%  Don't reorder the reviewer points; that'll mess up the automatic referencing!
%%
%%%%%

\begin{minipage}[b]{2.5in}
  Resubmission Cover Letter \\
  {\it PLoS Computational Biology}
\end{minipage}
\hfill
\begin{minipage}[b]{2.5in}
  \today
\end{minipage}
 
\vskip 2em
 
\noindent
{\bf To the Editor(s) -- }
 
\vskip 1em

We are pleased to submit a revision of our manuscript, 

\noindent \hspace{4em}
\begin{minipage}{3in}
\noindent
{\bf Sincerely,}

\vskip 2em

{\bf 
    Jerome Kelleher, Kevin Thornton, \\
    Jaime Ashander \emph{and} Peter Ralph \\
}\\
\end{minipage}

\vskip 4em

\pagebreak

%%%%%%%%%%%%%%
\reviewersection{AE}

\begin{quote}
    Both reviewers are clearly very enthusiastic about this manuscript, but both
    have made suggestions that require major revisions to address. Please be
    very careful in addressing the following points exhaustively:
    \begin{itemize}
        \item  Discuss the limitations imposed by Wright-Fisher assumptions (Rev 1) 
            and how the proposed approach may/may not be extended 
            to other demographic and population scenarios.

        \item  Improve the readability of the text and provide necessary background information (Rev 1)

        \item  Discuss implementation availability (Rev 2)
    \end{itemize}
\end{quote}

We thank the reviewers for helpful feedback, and respond in detail to their points below.
Something we did that should help address all these points
was to make more clear the main purpose of this paper:
to describe algorithms and data structures for tree sequences,
and how these are useful in simulation.
Our main goal is not to write a simulation package.
This is useful and important, because: 
(a) we provide a fully functional and flexible, well-documented API
so that developers can include this idea in their own simulator
(with no restrictions beyond conserved synteny of chromosomes); and
(b) we explain the data structures,
which are closely tied to the idea of a tree sequence,
and therefore important for even end users of a simulator to understand.
We describe and implement Wright--Fisher simulations not because of any limitation of our method,
but to keep things as simple as possible, only focusing on those aspects necessary 
for recording the tree sequence.

TODO: mention the tutorials?

%%%%%%%%%%%%%%
\reviewersection{1}

\begin{quote}
    Kelleher and colleagues present a novel method to efficiently record pedigree
    information in forward-time individual-based simulations of the Wright-Fisher
    process in a single large population of haploids. The method shows a large
    performance improvement compared to forward-time simulations of the same
    Wright-Fisher model without pedigree recording. The gain obtained with pedigree
    recording is due to the fact that mutations are generated post-hoc by placing
    them on the gene trees that resulted from the forward-time demographic process
    recorded as pedigree and cross-over information. The method for storing
    "succinct tree sequences" has been implemented in the coalescent-simulation
    software msprime and the authors here provide a Python API to use the data
    format previously developed in msprime in forward-time simulators. This
    represents a neat addition to the tool kit of population genetics simulations.
\end{quote}

We'd like to emphasize that although the tools for dealing with tree sequences
are currently bundled with the coalescent simulator, \msprime{},
this is for historical reasons only, 
and we are in the process of separating out these tools into a separate package,
to be called \tskit.
The format for storing tree sequences was developed for the coalescent simulator,
but the work we describe here generalized the data structure and algorithms
to allow it to be used in any simulator.
We have stated this more clearly in the text at \llname{msprime_rel}.

TODO: think about switching to \tskit{} in the paper?


\begin{point}{}
    Overall, I find the method very clever and appealing, and I am convinced that
    it will be well received by the community. However, nowadays, researchers
    generate and use genomic data in organisms other than humans and from wild
    populations that by far are not Wright-Fisher populations. How are you selling
    your method to that broader category of researchers? Compared to SPLATCHE,
    which simulates structured populations with gene flow, it may look like a
    regression (quantiNemo implemented something similar, but I don't think it was
    ever published).
    \emph{[\ldots]}
    As a developer, I am interested in the limitations and the constraints that may
    prevent me to implement your method to model populations that are not W-F.
    populations. In particular, you didn't address the question of structured
    populations with gene flow, or the effect of overlapping generations. Although
    developing an approach that is efficient in structured populations is probably
    out of scope here, giving a minimum of guidelines or just discussing the issue
    would greatly improve the scope of the manuscript.
\end{point}

\reply{
    We're sorry to have not made this more clear,
    but our goal was \emph{not} to make a faster WF simulator,
    but rather to describe a scheme that can be easily applied to \emph{any} simulator
    whatsoever, as long as it has the notion of a recombining chromosome.
    We have added a note to this effect before the proof-of-concept WF algorithm \llname{ll:not_wf};
    and substantially expanded the discussion of what steps need to be taken in general \revref.
    Also see the summarizing paragraph at \llname{ll:other_words}.
    The only thing that should stand in your way as a developer
    to implementing this is bookkeeping;
    and we've provided tools to make this bookkeeping easier
    (like the \texttt{sort tables} functionality,
    so that the simulator does not have to guarentee sortedness of the tables itself).
}


\begin{point}{}
    More generally, I found the ms very technical and addressed to a specialized
    audience. I would not assume, as you do, that readers are necessarily
    proficient in coalescent theory. Therefore, please improve the introduction by
    providing more information on: \emph{(points below)}
\end{point}

\reply{
    Thanks for the suggestion;
    since this paper is entirely about forwards simulation,
    we hope not to depend on coalescent theory at all in any crucial way.
    We have added some more explanation of the relationship to simplification
    at \revref;
    and have tried to better signpost the technical arguments at \llname{ll:coal_signpost}.
}


\begin{point}{}
    The idea of running simulations forward in time for the demography and then
    backward in time for the neutral genetics is by far not new and was already
    implemented in the software SPLATCHE (Ray, Currat, Foll, Excoffier,
    Bioinformatics 2010, 26: 2993-2994), although probably not as efficiently. The
    reference should be added.
\end{point}

\reply{
    We are assuming that you are drawing the analogy of tree sequence simplification 
    to the forward-backward scheme used by SPLATCHE.
    This is a good analogy, since simplification is tracing ancestry back through time
    as does the coalescent.
    However, we think it's fair to say that it differs fundamentally:
    the coalescent is a stochastic process that is dual to the forwards-time process,
    and relies on certain assumptions for correctness;
    whereas tree sequence simplification is deterministic (just ``bookkeeping'').
    We have added something about SPLATCHE2 to the introduction \revref.
}


\begin{point}{}
  What is the use of gene trees? What can be inferred/calculated from them that
  we cannot get from the endpoint genetic polymorphism data? (An example comes
very late in the ms, should be moved up.)
\end{point}

\reply{
    We agree this makes nice motivation,
    but think it is somewhat outside of the scope of this paper:
    for our purposes here, it suffices to explain that they help everything go faster.
    We've expanded a bit \revref.
}

\begin{point}{}
  Spell out once what is meant with $\theta = 4N_e\mu$ (and make clear that $\mu$ is
  the num loci $\times$ mutation rate)
\end{point}

\reply{
    We agree this was out of place; but rather than explain, we have removed this. \revref
}


\begin{point}{}
  Make clear how to transfer from haploid to diploid case (I guess by having
  about twice the data recorded).
\end{point}

\reply{
    We've tried to make this more clear. \revref
}

\begin{point}{}
  Figure 3 doesn't show the red stars mentioned in the main text (p8).
\end{point}

\reply{
    These are visible for us; there may have been a problem with PDF transparency;
    we've set opacity on these to 100\%; let us know if they are still invisible?
}



%%%%%%%%%%%%%%
\reviewersection{2}


\begin{quote}
    This manuscript presents algorithms that could very well change the way
    population genetic simulations are done. The current methods are either
    coalescent simulations (which build trees backwards in time, but for all
    intents and purposes do not allow selection) or forward simulations (which are
    as flexible as the user desires, but comes at the cost of being slow). The
    algorithms here are super creative solutions that allow the user to simulate a
    sequence tree forward in time. This approach can dramatically shrink the
    computational time needed to do large-scale simulations. The API proposed goes
    one step further. It is based on the msprime architecture, so a user could
    ostensibly seamlessly switch from coalescent to forward simulations very easily
    (or stitch the two together to generate ancient history using the coalescent
    and then forward simulate based on the output). This really could be a game
    changing paper for the population genetics world (as well as possibly others,
    as noted in the manuscript).

    I've thought quite a bit about how to record trees in a forward simulation, and
    could only come up with ways that drastically exploded the run time. The
    approach presented here is very creative.

    In addition, the use of the compressed format of the sequence tree is quite
    amazing. The 144k over the VCF equivalent is unbelievable.
\end{quote}

Thanks!


\begin{point}{}
    I only see one major criticism: accessibility. The authors do not provide a
    clear overview of how to obtain/install/use the algorithms presented. A
    ``proof-of-concept implementation'' is really not sufficient. In contrast,
    msprime has great documentation, but the current website does not seem to
    describe the forward simulation model (in contrast to what is suggested by the
    manuscript). \url{https://github.com/ashander/ftprime} does have installation
    instructions, but seems to be entirely geared toward integration with simuPOP.
    Most of the figures are generated with fwdpp, which are supposedly included
    here (\url{https://github.com/petrelharp/ftprime_ms}), but it is not at all clear how
    one would access the code given the current organization of this repository. It
    is unrealistic to expect all downstream users to code up the algorithms
    themselves (and would result in code that is not as well tested). The authors
    should have a more dedicated repository for users to access the software and
    start running their own simulations.
\end{point}

\reply{
    As the reviewer suggests, our main goal here is to provide easy access
    to the \emph{algorithm}, not to a forward simulation model --
    we aren't describing a forward simulation model, 
    but rather a way to speed up and get more information out of an existing one.
    But, is the algorithm accessible?
    An interface to our algorithms is available at the python level or at the C level.
    The python interface is documented through the \msprime{} documentation
    (but you need to select ``latest'': \url{https://msprime.readthedocs.io/en/latest/});
    and closely mirrors the underlying C library.
    We have also written documentation at \plrXXX LINK TO TUTORIALS}.
}

\begin{point}{}
Figure 1 shows that pedigree recording seems to scale poorly in N. Going from
N=1e3 to N=1e4 results in what seems like a $\sim 5 \times$ increase in runtime, but going
up to N=5e4 results in a $> 10 \times$ increase in runtime. Given your attention to
scaling, is this expected?
\end{point}

\reply{
    This is actually the fault of a quadratic algorithm for counting mutations
    under the hood in \fwdpp{}.
    In fact, some of us (KRT) are currently using ideas from tree sequences to fix this bottleneck.
}

%%%  Minor comments:

\begin{point}{}
The word "thanks” in the discussion of coalescent caveats is awkward.
\end{point}

\reply{
    Thanks; fixed. \revref
}

\begin{point}{}
The Appendices seem to be out of order with the text. The current order does
not seem any more logical.
\end{point}

\reply{
    Good point. Fixed.
}

\begin{point}{}
In Figure 1, it would potentially be helpful to see y-axis on a log scale so
that we could see scaling on the low end. Legend for this figure refers to
``right column of panels''. How many threads were used to obtain the dashed line?
\end{point}

\reply{
    Thanks for the suggestions.
    \plr{TODO: I've stuck in the "log" figure. What do you think?}
    \plr{Kevin: how many threads?}
}

\begin{point}{}
"its'" on page 11.
\end{point}

\reply{
    Thanks. Fixed, in two places.
}

\begin{point}{}
I appreciate the ``how does it perform in theory?'' section, but the final note
on the use of ``nedigree'' seems unnecessary. If you are going to use it (which I
do not recommend), you can justify it, otherwise drop it.
\end{point}

\reply{
    We have removed this section.
}

\begin{point}{}
What characteristics of each simulated mutation are stored? On page 10, it
suggests you store the node in which they first occur, the derived state, and
the genomic position. However, the selection coefficient and dominance
coefficient? Are there other features that can be stored (e.g., nucleotide
context)?
\end{point}

\reply{
    Good question;
    the answer is that anything can be stored (as arbitrary bytes);
    we've added a note about this. \revref
}

\begin{point}{}
The code for the Python implementation of simplify is included as supplementary
information. This is entirely unnecessary, as this code should be easily
accessible from the appropriate github repository.
\end{point}

\reply{
    Perhaps, but since github does not provide a permanent home,
    and a working implementation is, arguably, the most clear explanation of an algorithm,
    we'd prefer to keep this in as supplementary material.
}
