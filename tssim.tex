\documentclass{bioinfo}
\copyrightyear{2017} \pubyear{2017}

\newcommand{\plr}[1]{{\color{blue}\it #1}}

\access{Advance Access Publication Date: Day Month Year}
\appnotes{Application Notes}

\begin{document}
\firstpage{1}


% Bioinformatics:
% Application Notes (up to 2 pages; this is approx. 1300 words or 1000 words plus
% one figure) Applications Notes are short descriptions of novel software or new
% algorithm implementations, databases and network services (web servers, and
% interfaces). Software or data must be freely available to non-commercial users.
% Availability and Implementation must be clearly stated in the article. Authors
% must also ensure that the software is available for a full TWO YEARS following
% publication. Web services must not require mandatory registration by the user.
% Additional Supplementary data can be published online-only by the journal. This
% supplementary material should be referred to in the abstract of the Application
% Note. If describing software, the software should run under nearly all
% conditions on a wide range of machines. Web servers should not be browser
% specific. Application Notes must not describe trivial utilities, nor involve
% significant investment of time for the user to install. The name of the
% application should be included in the title.

\subtitle{Subject Section}

\title[Recording the pedigree]{ftprime: Recording the pedigree: efficient simulation of whole genomes}
\author{Corresponding Author\,$^{\text{\sfb 1,}*}$, Co-Author\,$^{\text{\sfb 2}}$ and Co-Author\,$^{\text{\sfb 2,}*}$}
\address{$^{\text{\sf 1}}$Department, Institution, City, Post Code, Country and \\
$^{\text{\sf 2}}$Department, Institution, City, Post Code,
Country.}

\corresp{$^\ast$To whom correspondence should be addressed.}

\history{Received on XXXXX; revised on XXXXX; accepted on XXXXX}

\editor{Associate Editor: XXXXXXX}

\abstract{
    To use genomic data for inference and prediction it is often necessary to obtain whole-genome information
    from individual-based simulations,
    but the computational burden of endowing each simulated individual with an entire genome can be substantial.
    In this note we describe how to both (a) dramatically reduce this burden and 
    (b) efficiently record the entire history of the population.
    We do this by simulating only those loci that may affect reproduction (those having non-neutral variants),
    and recording the entire history of genetic inheritance in an efficient data structure,
    on which neutral mutations can be quickly placed afterwards.
    \plr{make more clear data structure was already developed? refer to 'tree sequence' by name?}
    The algorithm is implemented in python,
    and is designed to be easily used by any forwards-time simulation software.
\textbf{Availability and implementation:} \\
\textbf{Contact:} \\
\textbf{Supplementary information:} Supplementary data are available at \textit{Bioinformatics} online.}

\maketitle

\section{Introduction}

% OUTLINE
% 1. motivate need for whole-genome fwd-time simulations; point out that we only recently have the computing power to do this
% 2. explain ARG: explain that for forwards-time only need selected loci as by defn all others can be put on afterwards
% 3. back-of-the-envelope calculation to compare cost of tracking whole genomes versus putting mutations on ARG
% 4. review something about your efficient methods for storing/traversing tree sequence
% 5. write down algorithm used in ftprime to produce msprime-ready coalescence records
% 6. comparison of speed in really simple case (but whole-chromosome) using simupop in both cases

In recent years the increase in computing power --
gradual only in comparison to the increase in our ability to sequence genomes --
has made it possible to simulate the evolution of whole genomes of realistically-sized populations.
This major milestone promises to remove the field's reliance on approximations of unknown applicability.
Thus far, much of our understanding of how genomes evolve
comes from models of one or a few linked loci \citep{ewens},
and/or results derived under the assumption of neutrality.
These are certainly important, but researchers today have widely differing views
on the practical importance of widespread selection \citep{encode,graur,backgroundsel}.
Thus far what little analytical progress has been made on models of ubiquitous selection
are hard to check due to a lack of realistic simulation.

Simulations of organisms endowed with functional, varying genomes
interacting with each other and spatially and temporally varying environments
will not only allow us to develop and test our understanding of evolution,
but also to produce quantitative predictions of population dynamics --
for instance, the spread of insecticide resistance and behavior modification of mosquitos.

The most commonly used simulation methods to date rely on neutrality,
as this assumption coupled with random mating allows the use of coalescent methods \citep{wakeley}.
These drastically reduce the amount of required computation,
because they exploit a time-reversal duality of such models
to \emph{only} simulate the portion of history that determines the genomes in the modern population.
How does this work?
Suppose that to the \emph{population pedigree} --
the entire history of parent-offspring relationships of an entire population going back to a remote time --
we add information encoding the genetic outcomes of each ancestral meiosis --
who inherited which parts of which parental chromosomes.
This embellished graph is known as the \emph{ancestral recombination graph}, or ARG \citep{griffiths}.
Combined with ancestral genotypes and the origins of new mutations,
it completely specifies the genomic sequence of any individual in the population at any time.
However, much less than the entire ARG is needed to specify relationships between any given set of samples --
only those portions of it from which those samples have actually inherited, back to their most recent common ancestors.
For instance, \plr{add diagram of simple example?}
The assumptions of coalescent theory -- random mating and neutrality --
imply that the stochastic law of this random set forms a Markov process looked at \emph{backwards} in time,
and can hence be simulated without reference to the unnecessary remainder.
Although this point was in principle known since \citet{griffithsmarjoram},
only recent algorithmic advances made it possible to actually simulate the process correctly,
without approximation, across whole genomes \citep{msprime}.

These efficient methods are not available if either key assumption of coalescent theory --
neutrality or random mating -- are broken.
(Extensions can be made for only a few loci under selection \citep{hudson,krone}.)
Forwards-time simulation, however, is burdened by recording and passing on entire genomes,
which typically have tens to hundreds of millions of varying sites.


\begin{methods}
\section{Methods and Implementation}

\end{methods}
\section{Results and Discussion}



\section{Conclusion}

\section*{Acknowledgements}

\section*{Funding}

\bibliographystyle{natbib}
\bibliography{refs.bib}

\end{document}
